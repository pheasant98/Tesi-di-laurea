%**************************************************************
% file contenente le impostazioni della tesi
%**************************************************************

%**************************************************************
% Frontespizio
%**************************************************************

% Autore
\newcommand{\myName}{Federico Perin}                                    
\newcommand{\myTitle}{Analisi, progettazione e sviluppo di un motore conversazionale per una piattaforma di gestione della forza lavoro}

% Tipo di tesi                   
\newcommand{\myDegree}{Tesi di laurea triennale}

% Università             
\newcommand{\myUni}{Università degli Studi di Padova}

% Facoltà       
\newcommand{\myFaculty}{Corso di Laurea in Informatica}

% Dipartimento
\newcommand{\myDepartment}{Dipartimento di Matematica "Tullio Levi-Civita"}

% Titolo del relatore
\newcommand{\profTitle}{Prof.}

% Relatore
\newcommand{\myProf}{Claudio Enrico Palazzi}

% Luogo
\newcommand{\myLocation}{Padova}

% Anno accademico
\newcommand{\myAA}{2019-2020}

% Data discussione
\newcommand{\myTime}{Settembre 2020}

\newcommand{\paginavuota}{\newpage\null\thispagestyle{empty}}

%**************************************************************
% Impostazioni di impaginazione
% see: http://wwwcdf.pd.infn.it/AppuntiLinux/a2547.htm
%**************************************************************

\setlength{\parindent}{14pt}   % larghezza rientro della prima riga
\setlength{\parskip}{0pt}   % distanza tra i paragrafi


%**************************************************************
% Impostazioni di biblatex
%**************************************************************
\bibliography{bibliografia} % database di biblatex 

\defbibheading{bibliography} {
    \cleardoublepage
    \phantomsection 
    \addcontentsline{toc}{chapter}{\bibname}
    \chapter*{\bibname\markboth{\bibname}{\bibname}}
}

\setlength\bibitemsep{1.5\itemsep} % spazio tra entry

\DeclareBibliographyCategory{opere}
\DeclareBibliographyCategory{web}

\addtocategory{opere}{womak:lean-thinking}
\addtocategory{web}{site:agile-manifesto}

\defbibheading{opere}{\section*{Riferimenti bibliografici}}
\defbibheading{web}{\section*{Siti Web consultati}}


%**************************************************************
% Impostazioni di caption
%**************************************************************
\captionsetup{
    tableposition=top,
    figureposition=bottom,
    font=small,
    format=hang,
    labelfont=bf
}

%**************************************************************
% Impostazioni di glossaries
%**************************************************************

%**************************************************************
% Acronimi
%**************************************************************
\renewcommand{\acronymname}{Acronimi e abbreviazioni}

\newacronym[description={\glslink{apig}{Application Program Interface}}]
    {api}{API}{Application Program Interface}
    
\newacronym[description={\glslink{AWMSg}{Advanced Workforce Management System}}]
{AWMS}{AWMS}{Advanced Workforce Management System}

\newacronym[description={\glslink{umlg}{Unified Modeling Language}}]
    {uml}{UML}{Unified Modeling Language}
\newacronym[description={Test End to End}]
{test e2e}{E2E}{test End to End}
\newacronym[description={Internazionalizzazione}]
{i18n}{i18n}{internazionalizzazione}

%**************************************************************
% Glossario
%**************************************************************
\renewcommand{\glossaryname}{Glossario}

\newglossaryentry{apig}
{
    name=\glslink{api}{API},
    text=Application Program Interface,
    sort=api,
    description={In informatica con il termine \emph{Application Programming Interface API} (ing. interfaccia di programmazione di un'applicazione) si indica ogni insieme di procedure disponibili al programmatore, di solito raggruppate a formare un set di strumenti specifici per l'espletamento di un determinato compito all'interno di un certo programma. La finalità è ottenere un'astrazione, di solito tra l'hardware e il programmatore o tra software a basso e quello ad alto livello semplificando così il lavoro di programmazione}
}

\newglossaryentry{umlg}
{
    name=\glslink{uml}{UML},
    text=UML,
    sort=uml,
    description={in ingegneria del software \emph{UML, Unified Modeling Language} (ing. linguaggio di modellazione unificato) è un linguaggio di modellazione e specifica basato sul paradigma object-oriented. L'\emph{UML} svolge un'importantissima funzione di ``lingua franca'' nella comunità della progettazione e programmazione a oggetti. Gran parte della letteratura di settore usa tale linguaggio per descrivere soluzioni analitiche e progettuali in modo sintetico e comprensibile a un vasto pubblico}
}

\newglossaryentry{AWMSg}
{
	name=\glslink{AWMS}{AWMS},
	text=Advanced Workforce Management System,
	sort=AWMS,
	description={È una soluzione software che utilizza algoritmi di \gls{machine learning}, per risolvere uno dei problemi cardine di un \gls{plant manager} ovvero, la pianificazione ottimale della forza lavoro che ha disposizione. L'obbiettivo principale della soluzione è quello di pianificare la persona giusta al posto giusto in base alle competenze tecniche possedute del lavoratore}
}

\newglossaryentry{Electrolux}
{
	name={Electrolux},
	text=Electrolux,
	sort=electrolux,
	description={Electrolux è una multinazionale svedese produttrice di elettrodomestici con sede a Stoccolma}
}

\newglossaryentry{machine learning}
{
	name={Machine learning},
	text=Machine learning,
	sort=machine learning,
	description={Nell'ambito dell'informatica, l'apprendimento automatico è una variante alla programmazione tradizionale nella quale in una macchina si predispone l'abilità di apprendere qualcosa dai dati in maniera autonoma, senza istruzioni esplicite}
}

\newglossaryentry{plant manager}
{
	name={Plant manager},
	text=Plant manager,
	sort=plant manager,
	description={Detto anche responsabile di stabilimento, è colui che presiede e organizza le operazioni quotidiane degli impianti di produzione aziendali, di cui deve assicurare il funzionamento ottimale ed efficiente. Si occupa dei lavoratori, assegnando funzioni e ruoli, definendo orari di lavoro e produzione, formando i neo assunti. Raccoglie e analizza i dati di produzione per trovare eventuali spazi di miglioramento. Si occupa della sicurezza dei lavoratori e quella degli impianti; monitora le apparecchiature di produzione e, in caso di necessità, della loro riparazione o sostituzione}
}

\newglossaryentry{bot}
{
	name={Bot},
	text=bot,
	sort=bot,
	description={È un software progettato per simulare una conversazione con un essere umano. Lo scopo principale di questi software è quello di simulare un comportamento umano e sono a volte definiti anche agenti intelligenti e vengono usati per vari scopi come la guida in linea, per rispondere alle FAQ degli utenti che accedono a un sito. In alcuni utilizzano sofisticati sistemi di elaborazione del linguaggio naturale, ma molti si limitano a eseguire la scansione delle parole chiave nella finestra di input e fornire una risposta con le parole chiave più corrispondenti}
}

\newglossaryentry{QR code}
{
	name={QR code},
	text=QR code,
	sort=qr-code,
	description={È un codice a barre bidimensionale (o codice 2D), ossia a matrice, composto da moduli neri disposti all'interno di uno schema bianco di forma quadrata, impiegato tipicamente per memorizzare informazioni generalmente destinate a essere lette tramite uno smartphone}
}

\newglossaryentry{WebSocket}
{
	name={WebSocket},
	text=WebSocket,
	sort=websocket,
	description={È una tecnologia web che fornisce canali di comunicazione a due direzioni cioè gli interlecutori possono sia inviare sia ricevere contemporaneamente attraverso una singola connessione TCP}
}

\newglossaryentry{SCRUM}
{
	name={SCRUM},
	text=SCRUM,
	sort=scrum,
	description={È un framework agile per la gestione del ciclo di sviluppo del software, iterativo ed incrementale, concepito per gestire progetti e prodotti software o applicazioni di sviluppo. Nel proprio manifesto prevede i seguenti punti che lo caratterizzano, le persone e le interazioni sono più importanti dei processi e dei strumenti, meglio avere da subito software funzionante che documentazione ampia, meglio una collaborazione con il cliente piuttosto che fare una negoziazione del contratto, essere in grando di rispondere ai cambiamenti piuttosto che rispettare un piano. 
	I progetti Scrum progrediscono attraverso una serie di sprint che hanno una durata massima di un mese. Negli sprint vengono decisi quali requisiti devono essere soddisfatti, e quindi successivamente, progettati, implementati e testati}
}
\newglossaryentry{framework}{
	name={Framework},
	text=framework,
	sort=framework,
	description={In informatica con il termine framework 
	si indica un insieme di elementi software che un programmatore può usare o modificare per realizzare un programma. Rappresenta un astrazione composta da elementi universali e riutilizzabili con lo scopo di facilitare lo sviluppo di un programma e di far applicare buone norme di programmazione. Inoltre, un framework può offrire programmi di supporto, librerie, compilatori e documentazione per l'utilizzo}
}
\newglossaryentry{iOS}{
	name={iOS},
	text=iOS,
	sort=iOS,
	description={È un sistema operativo \emph{mobile} sviluppato da Apple per iPhone, iPod touch e iPad. Le versioni principali di iOS vengono distribuite ogni anno. L'attuale versione, iOS 13, è stata distribuita al pubblico il 19 settembre 2019}
}
\newglossaryentry{Android}{
	name={Android},
	text=Android,
	sort=Android,
	description={È un sistema operativo per dispositivi \emph{mobile} sviluppato da Google e basato sul kernel Linux, progettato principalmente per \emph{smartphone} e \emph{tablet}, interfacce utente specializzate per televisori (Android TV), automobili (Android Auto), orologi da polso (Wear OS), occhiali (Google Glass). L'attuale versione è Android 11}
}
\newglossaryentry{architettura}{
	name={Architettura},
	text=Android,
	sort=Android,
	description={In informatica con il termine architettura, in questo caso intesa come architettura software, l'organizzazione fondamentale di un sistema, definita dai suoi componenti, dalle relazioni reciproche tra i componenti e con l'ambiente, e i principi che ne governano la progettazione e l'evoluzione}
}
\newglossaryentry{linguaggio di markup}{
	name={Linguaggio di markup},
	text=linguaggio di markup,
	sort=linguaggio di markup,
	description={In informatica con il termine linguaggio di markup, un gruppo di regole detti marcatori, attraverso le quali vengono descritti i meccanismi di rappresentazione di un testo}
}
\newglossaryentry{browser web}{
	name={Browser web},
	text=browser web,
	sort=browser web,
	description={In informatica si intende un'applicazione per l'acquisizione, la presentazione e la navigazione di risorse sul web. Permette la visualizzazione dei contenuti ipertestuali, e la riproduzione di contenuti multimediali. Tra i browser più popolari vi sono Google Chrome, Internet Explorer, Mozilla Firefox, Microsoft Edge, Safari, Opera}
}

\newglossaryentry{applicazione nativa}{
	name={Applicazione nativa},
	text=applicazione nativa,
	sort=applicazione nativa,
	description={In informatica si intende un'applicazione scritta e compilata per una specifica piattaforma utilizzando i linguaggi di programmazione e librerie supportati dal particolare sistema operativo \emph{mobile}}
}

\newglossaryentry{applicazione web mobile}{
	name={Applicazione web mobile},
	text=applicazione web mobile,
	sort=applicazione web mobile,
	description={In informatica si intende pagine web ottimizzate per dispositivi \emph{mobile} scritte utilizzando tecnologie web, in particolare HTML, JavaScript e CSS. Inoltre, le applicazioni web non possono accedere alle funzionalità del dispositivo ad'esempio la fotocamera}
}

\newglossaryentry{applicazione ibrida}{ %va a benzina ma anche con l'elettricità
	name={Applicazione ibrida},
	text=applicazione ibrida,
	sort=applicazione ibrida,
	description={In informatica si intende applicazioni sviluppate con tecnologie web e vengono eseguite localmente all’interno di un’applicazione nativa. Grazie a cio possono interagire con il dispositivo ad'esempio utilizzare la fotocamera}
}

\newglossaryentry{notifica push}{ 
	name={Notifica push},
	text=notifica push,
	sort=notifica pusha,
	description={In informatica si intende un tipo di messaggistica istantanea grazie alla quale il messaggio perviene al destinatario senza che questo debba effettuare un'operazione di scaricamento. Tale modalità è quella tipicamente usata da applicazioni come \emph{WhatsApp} o da servizi di sistemi operativi come \gls{Android}\ap{[g]}, oppure da numerose applicazioni derivate da siti web come, ad esempio, il servizio meteo o quello delle notizie}
}

\newglossaryentry{licenza MIT}{ 
	name={Licenza MIT},
	text=licenza MIT,
	sort=licenza MIT,
	description={La Licenza MIT è una licenza di software libero. È una licenza permissiva, cioè permette il riutilizzo nel software proprietario sotto la condizione che la licenza sia distribuita con tale software}
}

\newglossaryentry{JSON}{ 
	name=\glslink{JSON}{JSON},
	text=JSON,
	sort=JSON,
	description={ È un semplice formato adatto all'interscambio di dati fra applicazioni client/server. Risulta essere facile da comprendere e da scrivere per le persone mentre per le macchine risulta essere un formato leggero e veloce da analizzare}
}

\begin{comment}
\newglossaryentry{e2eg}
{
	name=\glslink{test e2e}{E2E},
	text=Test End to End,
	sort=test End to End,
	description={Con il termine test end-to-end (end-to-end testing) si intende quell’attività di testing dell’interfaccia grafica vista dagli utenti del programma dall’inizio fino alla fine. In altre parole rappresenta una metodologia utilizzata per verificare se il flusso di un’applicazione si sta comportando come progettato dall’inizio fino alla fine senza che vengano rilevati dei errori che andrebbero a inficiare sulla qualità dell’applicazione stessa}
}

Parole da aggiungere

http
pooling
client
server
Client/Server
SDK -> acronimo HTML e CSS http
open-source
JSON
API REST
\end{comment}
 % database di termini
\makeglossaries


%**************************************************************
% Impostazioni di graphicx
%**************************************************************
\graphicspath{{immagini/}} % cartella dove sono riposte le immagini


%**************************************************************
% Impostazioni di hyperref
%**************************************************************
\hypersetup{
    %hyperfootnotes=false,
    %pdfpagelabels,
    %draft,	% = elimina tutti i link (utile per stampe in bianco e nero)
    %colorlinks=true,
    %linktocpage=true,
    pdfstartpage=1,
    pdfstartview=FitV,
    % decommenta la riga seguente per avere link in nero (per esempio per la stampa in bianco e nero)
    colorlinks=false, linktocpage=false, pdfborder={0 0 0}, pdfstartpage=1, pdfstartview=FitV,
    breaklinks=true,
    pdfpagemode=UseNone,
    pageanchor=true,
    pdfpagemode=UseOutlines,
    plainpages=false,
    bookmarksnumbered,
    bookmarksopen=true,
    bookmarksopenlevel=1,
    hypertexnames=true,
    pdfhighlight=/O,
    %nesting=true,
    %frenchlinks,
    urlcolor=webbrown,
    linkcolor=RoyalBlue,
    citecolor=webgreen,
    %pagecolor=RoyalBlue,
    %urlcolor=Black, linkcolor=Black, citecolor=Black, %pagecolor=Black,
    pdftitle={\myTitle},
    pdfauthor={\textcopyright\ \myName, \myUni, \myFaculty},
    pdfsubject={},
    pdfkeywords={},
    pdfcreator={pdfLaTeX},
    pdfproducer={LaTeX}
}

%**************************************************************
% Impostazioni di itemize
%**************************************************************
\renewcommand{\labelitemi}{$\ast$}

%\renewcommand{\labelitemi}{$\bullet$}
%\renewcommand{\labelitemii}{$\cdot$}
%\renewcommand{\labelitemiii}{$\diamond$}
%\renewcommand{\labelitemiv}{$\ast$}


%**************************************************************
% Impostazioni di listings
%**************************************************************
\lstset{
    language=[LaTeX]Tex,%C++,
    keywordstyle=\color{RoyalBlue}, %\bfseries,
    basicstyle=\small\ttfamily,
    %identifierstyle=\color{NavyBlue},
    commentstyle=\color{Green}\ttfamily,
    stringstyle=\rmfamily,
    numbers=none, %left,%
    numberstyle=\scriptsize, %\tiny
    stepnumber=5,
    numbersep=8pt,
    showstringspaces=false,
    breaklines=true,
    frameround=ftff,
    frame=single
} 


%**************************************************************
% Impostazioni di xcolor
%**************************************************************
\definecolor{webgreen}{rgb}{0,.5,0}
\definecolor{webbrown}{rgb}{.6,0,0}
\definecolor{SchoolColor}{rgb}{0.71, 0, 0.106}
%%Colori per le tabelle
\definecolor{grigetto}{RGB}{234, 234, 234}
\definecolor{rossoep}{RGB}{164,60,59}
\definecolor{darkblue}{RGB}{59,77,95}
\definecolor{giallo}{RGB}{251,168,11}
\definecolor{verde}{RGB}{87,180,0}
\definecolor{grigio}{gray}{.7}
\definecolor{grigioChiaro}{gray}{.9}

\newcommand{\classdesc}[2]{\item[\textbf{#1:}] #2}
\newcommand{\intest}[1]{\multicolumn{1}{>{\columncolor{rossoep}}c}{\textbf{#1}}}
\newcolumntype{Z}{>{\centering\arraybackslash}X}
\renewcommand{\tabularxcolumn}[1]{>{\arraybackslash}m{#1}}


%**************************************************************
% Altro
%**************************************************************

\newcommand{\omissis}{[\dots\negthinspace]} % produce [...]

% eccezioni all'algoritmo di sillabazione
\hyphenation
{
    ma-cro-istru-zio-ne
    gi-ral-din
}

\newcommand{\sectionname}{sezione}
\addto\captionsitalian{\renewcommand{\figurename}{Figura}
                       \renewcommand{\tablename}{Tabella}}

\newcommand{\glsfirstoccur}{\ap{{[g]}}}

\newcommand{\intro}[1]{\emph{\textsf{#1}}}

%**************************************************************
% Environment per ``rischi''
%**************************************************************
\newcounter{riskcounter}                % define a counter
\setcounter{riskcounter}{0}             % set the counter to some initial value

%%%% Parameters
% #1: Title
\newenvironment{risk}[1]{
    \refstepcounter{riskcounter}        % increment counter
    \par \noindent                      % start new paragraph
    \textbf{\arabic{riskcounter} #1}   % display the title before the 
                                        % content of the environment is displayed 
}{
    \par\medskip
}

\newcommand{\riskname}{Rischio}

\newcommand{\riskdescription}[1]{\textbf{\\Descrizione:} #1}

\newcommand{\risksolution}[1]{\textbf{\\Soluzione:} #1}

%**************************************************************
% Environment per ``use case''
%**************************************************************
\newcounter{usecasecounter}             % define a counter
\setcounter{usecasecounter}{0}          % set the counter to some initial value

%%%% Parameters
% #1: ID
% #2: Nome
\newenvironment{usecase}[2]{
    \renewcommand{\theusecasecounter}{\usecasename #1}  % this is where the display of 
                                                        % the counter is overwritten/modified
    \refstepcounter{usecasecounter}             % increment counter
    \vspace{10pt}
    \par \noindent                              % start new paragraph
    {\large \textbf{\usecasename #1: #2}}       % display the title before the 
                                                % content of the environment is displayed 
    \medskip
}{
    \medskip
}

\newcommand{\usecasename}{UC}

\newcommand{\usecaseactors}[1]{\textbf{\\Attori Principali:} #1. \vspace{4pt}}
\newcommand{\usecasepre}[1]{\textbf{\\Precondizioni:} #1. \vspace{4pt}}
\newcommand{\usecasedesc}[1]{\textbf{\\Scenario Principale:} #1 \vspace{4pt}}
\newcommand{\usecasepost}[1]{\textbf{\\Postcondizioni:} #1. \vspace{4pt}}
\newcommand{\usecasealt}[1]{\textbf{\\Scenario Alternativo:} #1. \vspace{4pt}}
\newcommand{\usecaseest}[1]{\textbf{\\Estensioni:} #1 \vspace{4pt}}
\newcommand{\usecaseflow}[1]{\textbf{\\Flusso di Eventi:} #1 \vspace{4pt}}
\newcommand{\usecasegen}[1]{\textbf{\\Generalizzazione:} #1 \vspace{4pt}}

%**************************************************************
% Environment per ``namespace description''
%**************************************************************

\newenvironment{namespacedesc}{
    \vspace{10pt}
    \par \noindent                              % start new paragraph
    \begin{description} 
}{
    \end{description}
   % \medskip
}




\usepackage[explicit]{titlesec}
\usepackage{xcolor}
\usepackage{lipsum}% just to generate text
%\colorlet{myrulecolor}{black}
%\definecolor{myrulecolor}{RGB}{150,20,0}% define the color for the rules
\titleformat{\chapter}[display]
{\normalfont\scshape\Huge}
{\hspace*{-70pt}\textcolor{SchoolColor}{\textbf{\thechapter}} \textbf{\textcolor{SchoolColor}{ |}} ~#1}
{-15pt}
{\hspace*{-110pt}{\color{SchoolColor}\rule{\dimexpr\textwidth+80pt\relax}{3pt}}\Huge}
\titleformat{name=\chapter,numberless}[display]
{\normalfont\scshape\Huge}
{\hspace*{-70pt}#1}
{-15pt}
{\hspace*{-110pt}{\color{SchoolColor}\rule{\dimexpr\textwidth+80pt\relax}{3pt}}\Huge}
\titlespacing*{\chapter}{0pt}{0pt}{30pt}
