% !TEX encoding = UTF-8
% !TEX TS-program = pdflatex
% !TEX root = ../tesi.tex

%**************************************************************
% Sommario
%**************************************************************
\cleardoublepage
\phantomsection
\pdfbookmark{Sommario}{Sommario}
\begingroup
\let\clearpage\relax
\let\cleardoublepage\relax
\let\cleardoublepage\relax

\chapter*{Sommario}

Il presente documento descrive il lavoro svolto durante il periodo di stage, della durata di trecentoquattro ore, dal laureando Federico Perin presso l'azienda AzzurroDigitale S.r.l.
Lo scopo dello stage era quello di essere introdotto all'interno del progetto aziendale "Azzura.flow". Tale progetto prevede lo sviluppo di un \g{bot} denominato Azzurra, da integrare all'interno di una applicazione \emph{mobile}. Azzurra quindi, attraverso una \emph{chat} con l'utilizzatore umano, svolgerà il ruolo di assistente offrendo funzionalità di supporto, come ad'esempio informare il lavoratore sul suo piano di lavoro. \\
 Era richiesto come primo obbiettivo, acquisire le competenze tecniche richieste per poter contribuire allo sviluppo nel progetto attraverso lo studio e l'utilizzo di video lezioni offerte dalla piattaforma di \emph{e-learning} Udeny.\\
  In secondo luogo, veniva richiesto lo studio del funzionamento dell'\g{architettura} del sistema che permette l'esecuzione di Azzurra, in particolare il funzionamento dei metodi del motore conversazionale denominato Azzura Flow Engine. Una volta apprese le conoscenze necessarie, si richiedeva la progettazione e l'implementazione di alcuni flussi di conversazione per Azzurra. Era inoltre richiesto lo studio di un \emph{template-engine} per permettere il supporto di più lingue all'interno di Azzurra, e quindi è stato poi richiesto di implementare \emph{template} di testi in multi-lingua. \\
   Affiancato alle attività di implementazioni era richiesto, da buona prassi, effettuare attività di documentazioni sia riguardante il codice ma anche di scelte progettuali, e lo sviluppo di una \emph{test-suite} di \gls{test e2e} per l'applicazione \emph{mobile} e per il \g{front-end}, in modo da verificare il corretto funzionamento.




%\vfill
%
%\selectlanguage{english}
%\pdfbookmark{Abstract}{Abstract}
%\chapter*{Abstract}
%
%\selectlanguage{italian}

\endgroup			

\vfill

