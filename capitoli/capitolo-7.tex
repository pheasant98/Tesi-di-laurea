% !TEX encoding = UTF-8
% !TEX TS-program = pdflatex
% !TEX root = ../tesi.tex

%**************************************************************
\chapter{Conclusioni}
\label{cap:conclusioni}
%**************************************************************

%**************************************************************
\section{Consuntivo finale}
Rispetto al piano di lavoro pianificato esposto nella sezione \hyperref[cap:pianificazione]{§2.5}, sono avvenute delle modifiche e degli scostamenti. Ciononostante, il progetto di stage si è concluso soddisfacendo gli obbiettivi e i prodotti richieste e pianificati. Nonostante le ore totali pianificate inizialmente erano di 320 ore, tutte le attività pianificate si sono concluse con due giorni d'anticipo, perciò il totale di ore svoltesi è stato di 304. Ciò significa che la nona settimana non è stato svolta essendosi concluso prima lo stage. Nella prima settimane c'è stato uno scostamento di dodici ore questo perché lo studio approfondito delle tecnologie da utilizzare ha richiesto più tempo del previsto. Nella seconda settimana su decisione del tutor aziendale, si è spostato l'implementazione dei \emph{test} \gls{test e2e} per il \emph{mobile} nella sesta settimana applicando l'attività testando tutte le funzionalità dell'applicazione \emph{mobile} e non solo alcune funzionalità come era stato pianificato. Nella terza settimana sono state richieste quattro ore in più a causa di un \emph{bug} manifestatosi di QR Scanner, ciononostante è stato egregiamente gestito e risolto. Nella quarta e quinta settimana si sono risparmiate per ogni settimana otto ore. Nella sesta settimana si è aggiunto l'implementazione dei \emph{test} \gls{test e2e} per il \emph{mobile} questo perché Non è stato possibile effettuare l'implementazione delle \glslink{notifica push}{notifiche push}\textcolor{SchoolColor}{\ap{[g]}} a causa di esigenze aziendali cioè, l'azienda aveva l'esigenza di avere già subito implementate le \glslink{notifica push}{notifiche push}\textcolor{SchoolColor}{\ap{[g]}}, cosi sono state implementante dai membri dell'azienda. L'attività è stata sostituita dall'attività di documentazione delle \glslink{notifica push}{notifiche push}\textcolor{SchoolColor}{\ap{[g]}} oltre all'attività di \emph{testing}.
Nella ottava settimana sono state richieste meno ore di quanto pianificato risparmiando cosi otto ore.
Di seguito viene mostrata la tabella riportante il consuntivo finale del progetto di stage.

\begin{table}[h]%
	\rowcolors{2}{grigetto}{white}
	\renewcommand{\arraystretch}{1.7}
	\centering
	\begin{tabularx}{\textwidth}{X c c c}
		\hline	
		\rowcolor{heavenly}
		\intest{Attività} & \intest{Ore Pianificate} & \intest{Ore Effettive} & \intest{Scostamento}\\	
		\hline			
		Studio delle tecnologie, Angular 2+ e Ionic, da utilizzare durante lo stage & 24 & 36 & 12 \\
		
		Studio di componenti del dell'architettura di sistema di Azzurra, creazione di \emph{test} per la \emph{dashboard} di \gls{AWMS} e per l'applicazione \emph{mobile} & 40 & 40 & 0 \\
\end{tabularx} \hbox{}

\caption{Tabella riassuntiva del consultivo delle attività per il progetto di stage}
\end{table}
		
\begin{table}[h]%
	\rowcolors{2}{grigetto}{white}
	\renewcommand{\arraystretch}{1.7}
	\centering
	\begin{tabularx}{\textwidth}{X c c c}
		\hline	
		\rowcolor{heavenly}
		\intest{Attività} & \intest{Ore Pianificate} & \intest{Ore Effettive} & \intest{Scostamento}\\	
		\hline		
		Continuazione studio delle componenti del sistema di Azzurra e analisi, progettazione e implementazione di flussi conversazionali. & 40 & 44 & +4 \\
		
		Scritture di documentazione per le componenti di Azzurra. & 40 & 32 & -8\\
		
		Continuazione studio di altre componenti di \gls{AWMS}. & 40 & 32 & -8\\
		
		Documentazione delle componenti \gls{AWMS} e implementazione \glslink{notifica push}{notifiche push}\textcolor{SchoolColor}{\ap{[g]}}. & 40 & 48 & +8 \\
		
		Progettazione, implementazione e documentazione di \emph{template engine} multi-lingua. & 40 & 40 & 0 \\
		
		Studio della gestione dei comportamenti \emph{mobile} \emph{application} in condizioni di mancanza di connettività. & 40 & 32 & -8 \\
		
		Continuazione ottava settimana. & 16 & non svolte & - \\
		\hline
	\end{tabularx} \hbox{}
	
	\caption{Tabella riassuntiva del consultivo delle attività per il progetto di stage}
\end{table}%
%**************************************************************
\section{Raggiungimento degli obiettivi}

Al termine del progetto di stage sono stati raggiunti tutti gli obiettivi gli obbiettivi pianificati, validati dalla consegna dei prodotti attesi indicati nella sezione \hyperref[cap:prodotti]{§2.3}

Di seguito vengono riportati gli obbiettivi che fanno riferimento alla loro pianificazione descritta nella sezione \hyperref[cap:obbiettivi]{§2.2}.

\subsection*{Obbligatori}
\begin{itemize}
	\item \textbf{OB-1}: Raggiunto, Attraverso l'implementazione dei flussi conversazionali e della \emph{test-suite} per l'applicazione \emph{mobile} e per la \emph{dashboard} di \gls{AWMS} è stato dimostrato il raggiungimento dell'obbiettivo.
\end{itemize}
\subsection*{Desiderabili} 
\begin{itemize}
	\item \textbf{OD-1}: Raggiunto, durante l'analisi delle componenti utili per l'implementazione dei flussi conversazionali e della \emph{test-suite} per l'applicazione \emph{mobile} e per la \emph{dashboard} di \gls{AWMS} non è stato richiesto particolare aiuto al tutor aziendale lavorando in autonomia;
	\item \textbf{OD-2}: Raggiunto, durante la progettazione delle componenti per l'implementazione dei flussi conversazionali e della \emph{test-suite} per l'applicazione \emph{mobile} e per la \emph{dashboard} di \gls{AWMS} non è stato richiesto particolare aiuto al tutor aziendale lavorando in autonomia.
\end{itemize}

\subsection{Riepilogo}

Di seguito viene mostrata la tabella riportante il resoconto di tutti gli obiettivi con il
relativo stato alla data di fine stage.
\begin{table}[h]%
	\rowcolors{2}{grigetto}{white}
	\renewcommand{\arraystretch}{1.7}
	\centering
	\begin{tabularx}{\textwidth}{c X c}
		\hline	
		\rowcolor{heavenly}
		\intest{Codice} & \intest{Obbiettivo} & \intest{Esito} \\	
		\hline		
		OB-1 & Competenza nello sviluppo delle singole attività identificate con i linguaggi \gls{PHP} e Typescript. & Raggiunto \\
		
		OD-1 & Capacità autonoma di analisi delle singole attività delle soluzioni tecniche viste durante il progetto. & Raggiunto \\
		
		OD-2 & Capacità autonoma di progettazione delle singole attività delle soluzioni tecniche viste durante il progetto. &  Raggiunto \\
		\hline
\end{tabularx} \hbox{}

\caption{Tabella riassuntiva degli obbiettivi pianificati del progetto di stage}
\end{table}%
\section{Consegna dei prodotti}
Al termine del progetto di stage sono stati tutti i prodotti attesi 
Di seguito vengono riportati i prodotti consegnati nella \emph{repository} aziendale su GitLab e sulla piattaforma Confluence.

\begin{itemize}
	\item \textbf{Analisi tecnica}: È stato consegnata la documentazione dell'analisi tecnica del relativa alle componenti di \gls{AWMS} e di ciò che è stato prodotto durante lo stage;
	\item \textbf{Software}: Sono stati consegnati nella \emph{repository} aziendale su GitLab, i flussi conversazionali prodotti e la test-suite prodotta, i quali sono pronti per ad essere integrati con i prodotti dell'azienda.
\end{itemize}

\subsection{Riepilogo}

Di seguito viene mostrata la tabella riportante il resoconto di tutti i prodotti con il
relativo stato della consegna alla data di fine stage.
\clearpage
\begin{table}[h]%
	\rowcolors{2}{grigetto}{white}
	\renewcommand{\arraystretch}{1.7}
	\centering
	\begin{tabularx}{\textwidth}{X c}
		\hline	
		\rowcolor{heavenly}
		\intest{Prodotto} & \intest{Esito} \\	
		\hline		
		Descrizione dell’analisi svolta e soluzione identificata, sarà redatta sulla piattaforma documentale aziendale Confluence. & Consegnato \\
		Implementazione \emph{software} della soluzione identificata, redatta con l’IDE di sviluppo identificato per il progetto e depositata sul \emph{repository} GitLab di riferimento. & Consegnato \\
		\hline
	\end{tabularx} \hbox{}

\caption{Tabella riassuntiva dei prodotti pianificati del progetto di stage}
\end{table}%


%**************************************************************
\section{Conoscenze acquisite}
Lorem ipsum dolor sit amet, consectetur adipiscing elit. Proin ac luctus dui. Ut a eros sed dui faucibus maximus at feugiat nunc. Nullam vulputate diam tellus. Cras fermentum rhoncus lectus sit amet semper. In sodales, nisi vel consectetur aliquet, diam diam pulvinar dolor, et pellentesque justo dolor dictum enim. In sollicitudin nibh vel metus ultrices pretium. Aliquam a odio ac justo faucibus suscipit. Mauris luctus tellus id felis efficitur, sit amet blandit dolor semper. Aliquam erat volutpat. Phasellus tincidunt eu nibh vel tincidunt. Orci varius natoque penatibus et magnis dis parturient montes, nascetur ridiculus mus. Mauris a pellentesque elit. Sed vel auctor risus, non feugiat mauris. Aenean accumsan eros at vehicula elementum. Proin tincidunt porttitor volutpat.
%**************************************************************
\section{Valutazione personale}
Lorem ipsum dolor sit amet, consectetur adipiscing elit. Proin ac luctus dui. Ut a eros sed dui faucibus maximus at feugiat nunc. Nullam vulputate diam tellus. Cras fermentum rhoncus lectus sit amet semper. In sodales, nisi vel consectetur aliquet, diam diam pulvinar dolor, et pellentesque justo dolor dictum enim. In sollicitudin nibh vel metus ultrices pretium. Aliquam a odio ac justo faucibus suscipit. Mauris luctus tellus id felis efficitur, sit amet blandit dolor semper. Aliquam erat volutpat. Phasellus tincidunt eu nibh vel tincidunt. Orci varius natoque penatibus et magnis dis parturient montes, nascetur ridiculus mus. Mauris a pellentesque elit. Sed vel auctor risus, non feugiat mauris. Aenean accumsan eros at vehicula elementum. Proin tincidunt porttitor volutpat.
Lorem ipsum dolor sit amet, consectetur adipiscing elit. Proin ac luctus dui. Ut a eros sed dui faucibus maximus at feugiat nunc. Nullam vulputate diam tellus. Cras fermentum rhoncus lectus sit amet semper. In sodales, nisi vel consectetur aliquet, diam diam pulvinar dolor, et pellentesque justo dolor dictum enim. In sollicitudin nibh vel metus ultrices pretium. Aliquam a odio ac justo faucibus suscipit. Mauris luctus tellus id felis efficitur, sit amet blandit dolor semper. Aliquam erat volutpat. Phasellus tincidunt eu nibh vel tincidunt. Orci varius natoque penatibus et magnis dis parturient montes, nascetur ridiculus mus. Mauris a pellentesque elit. Sed vel auctor risus, non feugiat mauris. Aenean accumsan eros at vehicula elementum. Proin tincidunt porttitor volutpat.
Lorem ipsum dolor sit amet, consectetur adipiscing elit. Proin ac luctus dui. Ut a eros sed dui faucibus maximus at feugiat nunc. Nullam vulputate diam tellus. Cras fermentum rhoncus lectus sit amet semper. In sodales, nisi vel consectetur aliquet, diam diam pulvinar dolor, et pellentesque justo dolor dictum enim. In sollicitudin nibh vel metus ultrices pretium. Aliquam a odio ac justo faucibus suscipit. Mauris luctus tellus id felis efficitur, sit amet blandit dolor semper. Aliquam erat volutpat. Phasellus tincidunt eu nibh vel tincidunt. Orci varius natoque penatibus et magnis dis parturient montes, nascetur ridiculus mus. Mauris a pellentesque elit. Sed vel auctor risus, non feugiat mauris. Aenean accumsan eros at vehicula elementum. Proin tincidunt porttitor volutpat.