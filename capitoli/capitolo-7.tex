% !TEX encoding = UTF-8
% !TEX TS-program = pdflatex
% !TEX root = ../tesi.tex

%**************************************************************
\chapter{Conclusioni}
\label{cap:conclusioni}
%**************************************************************

%**************************************************************
\section{Consuntivo finale}
Rispetto al piano di lavoro esposto nella sezione \hyperref[cap:pianificazione]{§2.5}, sono state apportate delle modifiche. Ciononostante, il progetto di stage si è concluso soddisfacendo tutti gli obiettivi e i prodotti richiesti e pianificati nonostante le ore totali pianificate inizialmente fossero trecentoventi ed il totale effettivo ammonta a trecentoquattro: tutte le attività pianificate si sono concluse con due giorni d'anticipo previsti nella nona settimana.\\

Nelle attività della prima settimane c'è stato un aumento di dodici ore perché lo studio approfondito delle tecnologie da utilizzare ha richiesto più tempo del previsto. Per le attività della seconda settimana, su decisione del tutor aziendale, si è spostato l'implementazione dei \emph{test} \gls{test e2e} per il \emph{mobile} alla sesta settimana, ampliando l'attività con il \emph{testing} di tutte le funzionalità dell'applicazione \emph{mobile} e non solo alcune funzionalità come era stato pianificato inizialmente. Per le attività della terza settimana sono state necessarie quattro ore in più a causa di un \emph{bug} manifestatosi in QR Scanner, egregiamente gestito e risolto. Nelle attività della quarta e della quinta settimana si sono risparmiate per ogni settimana otto ore. Tra le attività della sesta settimana si è aggiunto l'implementazione dei \emph{test} \gls{test e2e} per il \emph{mobile} questo perché non è stato possibile effettuare l'implementazione delle \glslink{notifica push}{notifiche push}\textcolor{SchoolColor}{\ap{[g]}} a causa di esigenze aziendali cioè, l'azienda aveva l'esigenza di avere già subito implementate le \glslink{notifica push}{notifiche push}\textcolor{SchoolColor}{\ap{[g]}}, cosi sono state implementante dai membri dell'azienda. L'attività è stata sostituita dall'attività di documentazione delle \glslink{notifica push}{notifiche push}\textcolor{SchoolColor}{\ap{[g]}} oltre all'attività di \emph{testing}.
Per le attività della ottava settimana sono state richieste meno ore di quanto pianificato risparmiando cosi otto ore.\\

Di seguito viene mostrata la tabella riportante il consuntivo finale del progetto di stage.

\begin{table}[h]%
	\rowcolors{2}{grigetto}{white}
	\renewcommand{\arraystretch}{1.7}
	\centering
	\begin{tabularx}{\textwidth}{X c c c}
		\hline	
		\rowcolor{heavenly}
		\intest{Attività} & \intest{Ore Pianificate} & \intest{Ore Effettive} & \intest{Scostamento}\\	
		\hline			
		Studio delle tecnologie, Angular 2+ e Ionic, da utilizzare durante lo stage & 24 & 36 & 12 \\
		
		Studio di componenti del dell'architettura di sistema di Azzurra, creazione di \emph{test} per la \emph{dashboard} di \gls{AWMS} e per l'applicazione \emph{mobile} & 40 & 40 & 0 \\
\end{tabularx} \hbox{}

\caption{Tabella riassuntiva del consultivo delle attività per il progetto di stage}
\end{table}
		
\begin{table}[h]%
	\rowcolors{2}{grigetto}{white}
	\renewcommand{\arraystretch}{1.7}
	\centering
	\begin{tabularx}{\textwidth}{X c c c}
		\hline	
		\rowcolor{heavenly}
		\intest{Attività} & \intest{Ore Pianificate} & \intest{Ore Effettive} & \intest{Scostamento}\\	
		\hline		
		Continuazione studio delle componenti del sistema di Azzurra e analisi, progettazione e implementazione di flussi conversazionali. & 40 & 44 & +4 \\
		
		Scritture di documentazione per le componenti di Azzurra. & 40 & 32 & -8\\
		
		Continuazione studio di altre componenti di \gls{AWMS}. & 40 & 32 & -8\\
		
		Documentazione delle componenti \gls{AWMS} e implementazione \glslink{notifica push}{notifiche push}\textcolor{SchoolColor}{\ap{[g]}}. & 40 & 48 & +8 \\
		
		Progettazione, implementazione e documentazione di \emph{template engine} multi-lingua. & 40 & 40 & 0 \\
		
		Studio della gestione dei comportamenti \emph{mobile} \emph{application} in condizioni di mancanza di connettività. & 40 & 32 & -8 \\
		
		Continuazione ottava settimana. & 16 & non svolte & - \\
		\hline
	\end{tabularx} \hbox{}
	
	\caption{Tabella riassuntiva del consultivo delle attività per il progetto di stage}
\end{table}%
%**************************************************************
\section{Raggiungimento degli obiettivi}

Al termine del progetto di stage sono stati raggiunti tutti gli obiettivi pianificati, validati dalla consegna dei prodotti attesi indicati nella sezione \hyperref[cap:prodotti]{§2.3}.

Di seguito vengono riportati gli obiettivi che fanno riferimento alla loro pianificazione descritta nella sezione \hyperref[cap:obbiettivi]{§2.2}.

\subsection*{Obbligatori}
\begin{itemize}
	\item \textbf{OB-1}: Raggiunto. Attraverso l'implementazione dei flussi conversazionali e della \emph{test-suite} per l'applicazione \emph{mobile} e per la \emph{dashboard} di \gls{AWMS} è stato dimostrato il raggiungimento dell'obiettivo.
\end{itemize}
\subsection*{Desiderabili} 
\begin{itemize}
	\item \textbf{OD-1}: Raggiunto. Durante l'analisi delle componenti utili per l'implementazione dei flussi conversazionali e della \emph{test-suite} per l'applicazione \emph{mobile} e per la \emph{dashboard} di \gls{AWMS} non è stato richiesto particolare aiuto al tutor aziendale lavorando in autonomia;
	\item \textbf{OD-2}: Raggiunto. Durante la progettazione delle componenti per l'implementazione dei flussi conversazionali e della \emph{test-suite} per l'applicazione \emph{mobile} e per la \emph{dashboard} di \gls{AWMS} non è stato richiesto particolare aiuto al tutor aziendale lavorando in autonomia.
\end{itemize}

\subsection{Riepilogo}

Di seguito viene mostrata la tabella riportante il resoconto di tutti gli obiettivi con il
relativo stato alla data di fine stage.
\begin{table}[h]%
	\rowcolors{2}{grigetto}{white}
	\renewcommand{\arraystretch}{1.7}
	\centering
	\begin{tabularx}{\textwidth}{c X c}
		\hline	
		\rowcolor{heavenly}
		\intest{Codice} & \intest{Obbiettivo} & \intest{Esito} \\	
		\hline		
		OB-1 & Competenza nello sviluppo delle singole attività identificate con i linguaggi \gls{PHP} e Typescript. & Raggiunto. \\
		
		OD-1 & Capacità autonoma di analisi delle singole attività delle soluzioni tecniche viste durante il progetto. & Raggiunto. \\
		
		OD-2 & Capacità autonoma di progettazione delle singole attività delle soluzioni tecniche viste durante il progetto. &  Raggiunto. \\
		\hline
\end{tabularx} \hbox{}

\caption{Tabella riassuntiva degli obiettivi pianificati del progetto di stage}
\end{table}%
\section{Consegna dei prodotti}
Al termine del progetto di stage sono stati consegnati tutti i prodotti attesi.
Di seguito vengono riportati i prodotti consegnati nella \emph{repository} aziendale su GitLab e sulla piattaforma Confluence.

\begin{itemize}
	\item \textbf{Analisi tecnica}: È stato consegnata la documentazione dell'analisi tecnica del relativa alle componenti di \gls{AWMS} e di ciò che è stato prodotto durante lo stage;
	\item \textbf{Software}: I flussi conversazionali e la \emph{test-suite} che ho prodotto sono stati consegnati nella \emph{repository} aziendale su GitLab e sono pronti per ad essere integrati con i prodotti dell'azienda.
\end{itemize}

\subsection{Riepilogo}

Di seguito viene mostrata la tabella riportante il resoconto di tutti i prodotti con il
relativo stato della consegna alla data di fine stage.
\clearpage
\begin{table}[h]%
	\rowcolors{2}{grigetto}{white}
	\renewcommand{\arraystretch}{1.7}
	\centering
	\begin{tabularx}{\textwidth}{X c}
		\hline	
		\rowcolor{heavenly}
		\intest{Prodotto} & \intest{Esito} \\	
		\hline		
		Descrizione dell’analisi svolta e soluzione identificata, sarà redatta sulla piattaforma documentale aziendale Confluence. & Consegnato. \\
		Implementazione \emph{software} della soluzione identificata, redatta con l’IDE di sviluppo identificato per il progetto e depositata sul \emph{repository} GitLab di riferimento. & Consegnato. \\
		\hline
	\end{tabularx} \hbox{}

\caption{Tabella riassuntiva dei prodotti pianificati del progetto di stage}
\end{table}%


%**************************************************************
\section{Analisi retrospettiva}
Dopo essersi concluso il progetto di stage ho analizzato tutto il lavoro svolto prendendo consapevolezza di ciò che è stato fatto e di cosa ho acquisito e imparato da questa nuova esperienza. Inoltre ho compreso l’importanza dell'esperienza lavorativa nel settore della tecnologia che durante il mio percorso di studi non avevo colto.\\

In seguito verranno riportate le conoscenze e le competenze acquisite, le tecnologie e strumenti utilizzati ed una valutazione personale sull'esperienza di stage.
\subsection{Conoscenze acquisite}
Durante l'esperienza di stage è stato possibile apprendere nuove conoscenze e a raffinare quelle già in possesso. Nello specifico:
\begin{itemize}
	\item \g{framework} Angular2+: Angular è stato utilizzato per produrre un applicazione \emph{mobile} composta da una gerarchia di componenti che permettessero la gestione degli eventi sull'interfaccia grafica. Sul \g{framework} Angular avevo delle conoscenze pregresse già prima dell'esperienza di stage ma grazie a quest'ultima ho potuto ampliarle e migliorare l'utilizzo dei metodi offerti, soprattutto per quanto riguarda l'ottimizzazione delle prestazioni;
	\item \g{framework} Ionic: Ionic è stato utilizzato per sviluppare l'applicazione \emph{mobile} offrendo componenti grafiche ottimizzate per il \emph{mobile} e permettendo di utilizzare Angular e Cordova insieme. Ho perciò appreso questo nuovo \g{framework} utile per poter sviluppare applicazioni \emph{mobile};
	\item  \g{framework} Apache Cordova è stato utilizzato per sviluppare un'applicazione \emph{mobile} con tecnologie web come \gls{HTML}, \gls{CSS} e JavaScript e quindi poter sviluppare un' \g{applicazione ibrida} dando accesso anche ai sensori del dispositivo \emph{mobile}.
	\item \emph{template-engine}: Ho utilizzato il \emph{template-engine}, nello specifico \emph{Handlebars}, per definire \emph{template} di testi in vari lingue. Grazie all'esperienza di stage ho potuto conoscere il concetto di \emph{template-engine} e di capirne le grandi potenzialità;
	\item \emph{test} \gls{test e2e}: I \emph{test} \gls{test e2e} svolti durante il progetto di stage mi hanno permesso di ampliare le mie conoscenze sulla qualità del \emph{software} scoprendo e mettendo in pratica un nuovo tipo di \emph{testing} che prima non conoscevo.
\end{itemize}
\subsection{Competenze acquisite}
Grazie all'esperienza di stage affrontata, sono maturato dal punto di vista professionale e ho acquisito numerose nuove competenze. Nello specifico:
\begin{itemize}
	\item Sviluppo di applicazioni \emph{mobile} ibride: Durante il progetto di stage è stato richiesto di implementare alcune funzionalità vedi \hyperref[cap:flussi di conversazione]{§5} per l'\g{applicazione ibrida} \gls{AWMS} Azzurra. Ho quindi imparato ad utilizzare le tecnologie per l'implementazione e compreso i vantaggi dati dallo sviluppo un'\g{applicazione ibrida} rispetto a un'\g{applicazione nativa};
	\item Metodologia di lavoro \g{SCRUM}: Nell'esperienza di stage sono entrato in contatto con il \g{framework} agile per la gestione dei progetti \emph{software} \g{SCRUM} utilizzato dall'azienda che mi ha ospitato. È stato perciò formativo vedere applicato questo modello che avevo studiato in modo teorico nei corsi di Ingegneria del Software e Tecnologie Open-Source.
	\item Spirito di imprenditorialità: Durante l'esperienza di stage ho avuto la possibilità di entrare in contatto con l'attività di \emph{business} dell'azienda ospitante. Infatti è incentivato da parte dell'azienda che ogni componente del team esponesse idee su nuove funzionalità da aggiungere ai prodotti già esistenti o idee per nuovi prodotti futuri da sviluppare.
	% Per esempio propore di produre un film porno è un esempio di proposta per nuovi prodotti futuri.
	 Ho perciò potuto capire quali opportunità possano esserci nel mondo del lavoro nell'ambito dell'informatica ma soprattutto in quello della digitalizzazione delle aziende.
\end{itemize}
\subsection{Tecnologie e strumenti utilizzati}
Nelle attività di stage è stato necessario utilizzare nuove tecnologie e strumenti. Nello specifico si è utilizzato:
\begin{itemize}
	\item GitLab: È stato utilizzato per gestire il versionamento dei flussi conversazionali e della \emph{test-suite} sviluppati mantenendo lo storico di tutte le modifiche effettuate;
	\item Jira Software: È stato utilizzato per assegnarmi i vari \emph{task} pianificati durante lo stage;
	\item Jira Confluence: È stato utilizzato per redigere e consegnare la documentazione richiesta dal tutor aziendale;
	\item Ionic Angular e Cordova: Durante lo stage sono stati per sviluppare alcune funzionalità per un'\g{applicazione ibrida};
	\item Selenium, Protractor, Cucumber e Appium: Sono stati utilizzati per sviluppare i \emph{test} \gls{test e2e}.
\end{itemize}
%**************************************************************
\subsection{Valutazione personale}
Valutando questa nuova esperienza fatta, mi ritengo soddisfatto del percorso fatto durante lo stage. Grazie a questa nuova esperienza ho potuto acquisire nuove conosce come Ionic e Cordova che mi hanno permesso di apprendere un modo alternativo più veloce e semplice di costruire un’applicazione \emph{mobile} rispetto allo sviluppo con linguaggi nativi. Ho arricchito le mie conoscenze in Angular, soprattutto grazie ai colleghi con cui sono stato affiancato e ho imparato soluzioni migliori a quelle che già conoscevo. Ho inoltre imparato che cosa sono, come implementarli e come funzionano i \emph{test} \gls{test e2e}.\\
Infine mi è stata data l'opportunità di capire com'è il mondo del lavoro nell'ambito dell'informatica e di passare dalla teoria alla pratica per quanto riguarda la metodologia di lavoro \g{SCRUM}.
Nonostante le restrizioni dovute al COVID-19 ho avuto la fortuna di lavorare in presenza presso l'azienda, rispettando le norme sanitarie negli uffici dell'azienda AzzurroDigitale e trovando un ambiente ospitale e confortevole dove poter lavorare con tranquillità.
% Inoltre l'ambiente di lavoro era bello perchè c'erano tante tipe che faccevano le consulenti per vendere AWMS in giro
Durante lo stage sono stato affiancato non solo dal tutor aziendale Carlo Davanzo ma anche dai membri del team a cui Carlo fa capo. Ho perciò lavorato insieme a persone molto disponibili, nonostante i loro impegni lavorativi e molto preparate e simpatiche con cui è stato un piacere lavorare. \\

Sono infine stato soddisfatto dei risultati ottenuti sui quali ho fatto delle considerazione precedentemente esposte nelle sezioni \hyperref[cap:cons1]{§5.5} \hyperref[cap:cons2]{§6.5}.
Ritengo perciò che il progetto di stage da me sostenuto sia stato molto positivo e istruttivo per i risultati ottenuti, per le conoscenze e competenze acquisite e per i rapporti stretti con il personale dell'azienda che mi ha permesso di effettuare il tirocinio formativo.
