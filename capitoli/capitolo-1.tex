% !TEX encoding = UTF-8
% !TEX TS-program = pdflatex
% !TEX root = ../tesi.tex

%**************************************************************
\chapter{Introduzione}
\label{cap:introduzione}
%**************************************************************

\section{Convenzioni tipografiche}
Nella stesura del presente documento, sono state adottate le seguenti convenzioni tipografiche:
\begin{itemize}
	\item gli acronimi, le abbreviazioni e i termini ambigui o di uso non comune menzionati vengono definiti nel glossario, situato alla fine del documento;
	\item per i termini riportati nel glossario viene utilizzata la seguente nomenclatura: \emph{parola}\textcolor{SchoolColor}{\ap{[g]}};
	\item i termini in lingua straniera o facenti parti del gergo tecnico sono evidenziati con il carattere \emph{corsivo}.
\end{itemize}

%**************************************************************
\section{L'azienda AzzurroDigitale S.r.l}

Lo stage è stato svolto nell'azienda AzzurroDigitale S.r.l. situata nella zona industriale di Padova. AzzurroDigitale nasce nel 2015 quando tre giovani padovani (Carlo Pasqualetto, Jacopo Pertile e Antonio Fornari) fondarono la \emph{startup}, puntando fortemente nelle nuove emergenti tecnologie che il mercato offriva. Il primo cliente fu l'azienda \g{Electrolux} con la quale, grazie a una forte attività collaborativa, fu sviluppata una piattaforma per la gestione della forza lavoro denominata \gls{AWMS} che tutt’ora continua a ricevere miglioramenti. Dopo il successo ottenuto dalla collaborazione con \g{Electrolux}, i fondatori capirono che il mercato delle aziende manifatturiere è la nicchia sulla quale potevano puntare, soprattutto grazie al momento storico della \emph{digital transformation}. 

\begin{figure}[h]
	\begin{center}
		\includegraphics[scale=0.5]{Logo_azzurrodigite.png}
			\caption{Logo di AzzurroDigitale}
	\end{center}
\end{figure}
\pagebreak

Oggi AzzurroDigitale offre servizi di \emph{industrial digital transformation}, \emph{workforce management} e \emph{people empowerment}, con l'obiettivo comune di aiutare le aziende manifatturiere a migliorare ed implementare i loro processi grazie alle tecnologie, non intese come sostitutive all’uomo, ma come mezzi che abilitano le persone a lavorare nel miglior modo possibile massimizzando lo sforzo lavorativo.\\

%**************************************************************
\section{L'idea}

%Introduzione all'idea dello stage.
\subsection{Il contesto applicativo}
L'azienda AzzurroDigitale offre come principale servizio la piattaforma di gestione forza lavoro denominata \gls{AWMS}.\\
	\begin{figure}[!h] 
		\begin{center}
			\includegraphics[scale=0.4]{logoAWMS.jpg}
			\caption{Logo di AWMS}
		\end{center}
	\end{figure}

\gls{AWMS} è una soluzione \emph{software} che utilizza algoritmi di \g{machine learning} per risolvere uno i problemi cardine di un \g{plant manager} ovvero la pianificazione ottimale della forza lavoro che ha disposizione. L'obiettivo principale di tale soluzione è stabilire la persona giusta al posto giusto, in base alle competenze tecniche possedute del lavoratore, attraverso una pianificazione. Per permettere questo funzionamento, la piattaforma estrae dati sui lavoratori da \g{database} interni all'azienda che ha acquistato la soluzione, che ne descrivono le competenze possedute. Viene perciò registrato uno storico delle mansioni svolte per ogni lavoratore aggiornandolo nel tempo. Perciò, in base ai dati estratti dalla piattaforma, viene scelto il miglior candidato per un determinato compito. \\ \gls{AWMS} offre quindi la possibilità di pianificare il lavoro per il giorno successivo ma anche di gestire situazioni impreviste come ad esempio l'assenza di un lavoratore.

\subsection{Il progetto Azzurra.flow}

Il progetto Azzurra.flow nasce dalla esigenza, da parte dell'azienda AzzurroDigitale, di offrire un prodotto completo per tutti i soggetti coinvolti nelle attività lavorative. Con \gls{AWMS} si ha uno strumento che supporta i \emph{team leader} o i \g{plant manager} nella loro pianificazione del lavoro ma non si ha nessun strumento che supporti il lavoratore. Da questa mancanza nasce perciò il progetto "Azzurra.flow". Esso consiste nel creare un \g{bot} denominato Azzurra, inserito in un’applicazione \emph{mobile}, che permette di offrire le seguenti funzionalità utili all'utente:
\begin{itemize}
	\item Visualizzare il proprio turno di lavoro;
	\item Visualizzare i propri permessi lavorativi o richiederne di nuovi;
	\item Visualizzare avvisi da parte dell'azienda;
	\item Sapere qual'è il menu del giorno della mensa aziendale;
	\item Poter effettuare prenotazioni di un posto a sedere in una sala riunioni e visualizzare le proprie prenotazioni utilizzando un scannerizzatore \g{QR code} per riscattare il posto prenotato.
\end{itemize}
\begin{figure}[!h] 
	\begin{center}
		\includegraphics[scale=0.25]{azzurra.png}
		\caption{Logo del bot Azzurra}
	\end{center}
\end{figure}
Il progetto include non solo lo sviluppo dell'applicazione \emph{mobile} con Azzurra ma un motore conversazionale denominato Azzurra Flow Engine, in grado di poter generare una conversazione con il lavoratore, attraverso l'interpretazione dei flussi di conversazione, anche essi da sviluppare, che indicano quali azioni deve fare Azzurra. Questi flussi devono essere memorizzati in un preciso posto e a questo proposito è stato progettato che sia un \g{database} contenuto nella nuova componente Azzurra.io con il compito non solo di tenere memorizzati i flussi conversazionili esistenti e di inviarli a Azzurra quando li richiede, ma anche di fare da tramite tra l'applicazione con all'interno Azzurra e \gls{AWMS}, tutto attraverso una comunicazione tramite \g{WebSocket}.
%**************************************************************
\section{Organizzazione del testo}
Il capitolo corrente è l'introduzione del documento, dove si è spiegato brevemente l'ambito di lavoro e il progetto sul quale si è svolto lo stage.\\
In seguito il documento sarà organizzato con la seguente struttura:
\begin{description}
    
    \item[{\hyperref[cap:descrizione-stage]{Il secondo capitolo}}] descrive in modo dettagliato lo stage svolto, indicandone obiettivi, prodotti attesi, pianificazione delle attività, strumenti e tecnologie utilizzate e motivazioni personali.
    
    \item[{\hyperref[cap:archittettura del sistema AWMS]{Il terzo capitolo}}] illustra l'\g{architettura} del sistema \gls{AWMS} che permette il funzionamento di Azzurra. Vengono qui descritte le componenti dell'architettura e le varie operazioni tra essi.
    
    \item[{\hyperref[cap:flow engine]{Il quarto capitolo}}] approfondisce il funzionamento del motore conversazionale di Azzurra indicando come avviene una conversazione tra Azzurra e l'utente.
    
    \item[{\hyperref[cap:flussi di conversazione]{Il quinto capitolo}}] descrive il lavoro di analisi, progettazione e implementazione dei flussi conversazionali per Azzurra.
    
    \item[{\hyperref[cap:test]{Il sesto capitolo}}] descrive le tecnologie utilizzate per costruire una test-suite per Azzurra ed espone il piano di test che è stato stabilito, inserendo i risultati ottenuti.
    
    \item[{\hyperref[cap:conclusioni]{Il settimo capitolo}}] rappresenta la conclusione del documento in cui è proposto un riepilogo del lavoro svolto durante lo stage, degli obiettivi raggiunti ed infine una valutazione personale sull'esperienza di stage.
\end{description}
