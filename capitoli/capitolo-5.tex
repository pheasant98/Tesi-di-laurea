% !TEX encoding = UTF-8
% !TEX TS-program = pdflatex
% !TEX root = ../tesi.tex

%**************************************************************
\chapter{Flussi conversazionali prodotti}
\label{cap:flussi di conversazione}
%**************************************************************

\intro{In questo capitolo verrà descritto il lavoro che è stato fatto di analisi, progettazione e implementazione dei flussi conversazioniali per Azzurra creati durante lo stage.}\\

%**************************************************************
\section{Analisi dei requisiti}
\subsection{Descrizione del problema}
	Durante lo stage è stato deciso, di comune accordo con il tutor aziendale, di costruire due flussi conversazionali per il bot Azzurra, nello specifico:
	\begin{itemize}
		\item \textbf{DeskBooking}: Questo flusso conversazionale consiste nel gestire le prenotazioni di un posto a sedere. Deve esserci la possibilità di richiedere una nuova prenotazione, visualizzare la lista delle proprie prenotazioni e infine, visto che è richiesto, che per riscattare il posto a sedere, quando si sta per iniziare a usufruire del posto si deve scannerizzare un QR code, messo nel posto a sedere, per poter verificare se chi ha scansionato il QR code ha veramente diritto a usufruire del posto, c'è perciò bisogno di integrare un lettore di QR code in Azzurra per poter fare il controllo. Perciò, si deve poter aprire la fotocamera, scannerizzare il QR code che verrà usato per controllare se il lavoratore può usufruire del posto dopo il controllo, comunicare l'esito della verifica al lavoratore;
		\item \textbf{Planning}: Questo flusso conversazionale consiste nel far visualizzare al lavoratore il lavoro che deve svolgere. Deve esserci la possibilità di richiedere la visualizzazione del lavoro pianificato di uno specifico giorno oppure la possibilità di vedere il lavoro pianificato per tutta la settimana;
	\end{itemize}
\subsection{Requisiti}
Ogni requisito sarà strutturato come segue:
\begin{itemize}
	\item Identificativo: \textbf{R[Importanza][Tipologia][Codice]}\\
	Dove:
	\begin{itemize}
		\item \textbf{Importanza:}
		\begin{itemize}
			\item \textbf{1}: Requisito obbligatorio, ovvero irrinunciabile per almeno uno degli stakeholder
			\item \textbf{2}: Requisito desiderabile, ovvero non strettamente necessario ma che porta valore aggiunto riconoscibile
			\item \textbf{3}: Requisito opzionale, ovvero relativamente utile oppure contrattabile più avanti nel progetto
		\end{itemize}
		\item \textbf{Tipologia:}
		\begin{itemize}
			\item \textbf{F}: Funzionale, definisce una funzione di un sistema di uno o più dei suoi componenti
			\item \textbf{Q}: Qualitativo, definisce un requisito per garantire la qualità per un certo aspetto del progetto
			\item \textbf{P}: Prestazionale, definisce un requisito che garantisce efficienza prestazionale nel prodotto
			\item \textbf{V}: Vincolo, definisce un requisito volto a far rispettare un dato vincolo
		\end{itemize}
		\item \textbf{Codice:} Viene utilizzato per identificare univocamente il requisito tramite un numero progressivo\\
	\end{itemize}
\end{itemize}
Dopo un analisi del problema sono stati individuati i seguenti requisiti
\begin{table}[h]%
	\rowcolors{2}{grigetto}{white}
	\centering
	\begin{tabularx}{\textwidth}{|c|X|}
		\hline	
		\rowcolor{giallo}
		\intest{Codice} &  \intest{Descrizione} \\	
		\hline			
		R1F1 & Il lavoratore deve poter accedere alla funzionalità di prenotazione posto.\\
		R1F2 & Il lavoratore deve poter inserire una nuova prenotazione di un posto a sedere.\\
		R1F3 & Il lavoratore deve poter visualizzare le sue prenotazione.\\
		R1F4 & Il lavoratore deve poter scansionare il QR code per poter usufruire del posto prenotato.\\
		R1F5 & Il lavoratore deve poter inserire la data in cui vuole prenotare il posto a sedere se disponibile.\\
		R1F6 & Il lavoratore deve poter inserire l'ora di inizio della prenotazione che desidera.\\
		R1F7 & Il lavoratore deve poter inserire l'ora di terminazione della prenotazione che desidera.\\
		R1F8 & Il lavoratore deve poter inserire la stanza del posto a sedere che desidera prenotare.\\
		R1F9 & Il lavoratore deve poter inserire il posto a sedere che desidera prenotare se disponibile.\\
		R1F10 & Il lavoratore deve poter visualizzare il messaggio di conferma se la prenotazione del posto a sedere è andata a buon fine.\\
		R1F11 & Il lavoratore deve poter visualizzare il messaggio d'errore se non è stato possibile inserire la nuova prenotazione.\\
		R1F12 & Il lavoratore deve poter visualizzare le sue prenotazione del giorno corrente.\\
		R1F13 & Il lavoratore deve poter visualizzare le sue prenotazione del giorno successivo.\\
		R1F14 & Il lavoratore deve poter visualizzare le sue prenotazione di uno specifico giorno.\\
		R1F15 & Il lavoratore deve poter inserire la data del giorno in cui vuole vedere le prenotazioni.\\
		R1F15 & Il lavoratore, per ogni prenotazione, deve poter visualizzare l'ora di inizio della prenotazione.\\
		R1F17 & Il lavoratore, per ogni prenotazione, deve poter visualizzare l'ora di terminazione della prenotazione.\\	
	\hline	
	\end{tabularx} \hbox{}
\caption{Tabella del tracciamento dei requisiti}
\end{table}%
\\

\begin{table}[h]%
	\rowcolors{2}{grigetto}{white}
	\centering
	\begin{tabularx}{\textwidth}{|c|X|}
		\hline		
		\rowcolor{giallo}
		\intest{Codice} &  \intest{Descrizione} \\	
		\hline	
		R1F18 & Il lavoratore, per ogni prenotazione, deve poter visualizzare la stanza della prenotazione.\\
		R1F19 & Il lavoratore, per ogni prenotazione, deve poter visualizzare il posto della prenotazione.\\	
		R1F20 & Il lavoratore, dopo avere scannerizzato il QR code del posto a sedere, deve ricevere un messaggio di conferma se può usufruire del posto a sedere.\\
		R1F21 & Il lavoratore, dopo avere scannerizzato il QR code del posto a sedere, deve ricevere un messaggio d'errore se non può usufruire del posto a sedere.\\
		R1F22 & Il lavoratore deve poter visualizzazione la pianificazione di uno specifico giorno.\\
		R1F23 & Il lavoratore deve poter visualizzazione la pianificazione della settimana corrente.\\
		R1F24 & Il lavoratore deve poter inserire la data del giorno in cui vuole vederne la pianificazione del lavoro a lui assegnato.\\
		R1F25 & Il lavoratore deve poter visualizzare la data del giorno del lavoro pianificato.\\
		R1F26 & Il lavoratore deve poter visualizzare l'ora d'inizio del lavoro pianificato.\\
		R1F27 & Il lavoratore deve poter visualizzare l'ora di terminazione del lavoro pianificato.\\
		R1F28 & Il lavoratore deve poter visualizzare il lavoro che è stato pianificato per essere svolto.\\
		R1V1 & Per implementare i flussi conversazionali devono essere usati Angular e Ionic.\\
		R1V2 & Per gestire la fotocamera per la lettura del QR code deve essere usato il plugin di Cordova, QR Scanner.\\

		\hline	
\end{tabularx} \hbox{}
\caption{Tabella del tracciamento dei requisiti}
\end{table}%
\section{Progettazione}
Dopo aver individuato i requisiti che descrivono i flussi da costruire, si è passati alla progettazione dei flussi. Utilizzando la struttura dei flussi conversazionali spiegata nel precedente capitolo, si è potuto progettare i due flussi inserendo i blocchi conversazionali in unico diagramma per ogni flusso. Cosi facendo si è creato un diagramma per ogni flusso che ne descrive i passi che devono essere fatti dal bot Azzurra durante la conversazione. Successivamente si sono progettati i metodi da aggiungere ai metodi già esistenti del Flow Engine, in modo che possano essere interpretare anche loro dal Flow Engine.


