
%**************************************************************
% Acronimi
%**************************************************************
\renewcommand{\acronymname}{Acronimi e abbreviazioni}

\newacronym[description={\glslink{apig}{Application Program Interface}}]
    {api}{API}{Application Program Interface}
    
\newacronym[description={\glslink{AWMSg}{Advanced Workforce Management System}}]
{AWMS}{AWMS}{Advanced Workforce Management System}

\newacronym[description={\glslink{umlg}{Unified Modeling Language}}]
    {uml}{UML}{Unified Modeling Language}
\newacronym[description={Test End to End}]
{test e2e}{E2E}{test End to End}

%**************************************************************
% Glossario
%**************************************************************
\renewcommand{\glossaryname}{Glossario}

\newglossaryentry{apig}
{
    name=\glslink{api}{API},
    text=Application Program Interface,
    sort=api,
    description={In informatica con il termine \emph{Application Programming Interface API} (ing. interfaccia di programmazione di un'applicazione) si indica ogni insieme di procedure disponibili al programmatore, di solito raggruppate a formare un set di strumenti specifici per l'espletamento di un determinato compito all'interno di un certo programma. La finalità è ottenere un'astrazione, di solito tra l'hardware e il programmatore o tra software a basso e quello ad alto livello semplificando così il lavoro di programmazione}
}

\newglossaryentry{umlg}
{
    name=\glslink{uml}{UML},
    text=UML,
    sort=uml,
    description={in ingegneria del software \emph{UML, Unified Modeling Language} (ing. linguaggio di modellazione unificato) è un linguaggio di modellazione e specifica basato sul paradigma object-oriented. L'\emph{UML} svolge un'importantissima funzione di ``lingua franca'' nella comunità della progettazione e programmazione a oggetti. Gran parte della letteratura di settore usa tale linguaggio per descrivere soluzioni analitiche e progettuali in modo sintetico e comprensibile a un vasto pubblico}
}

\newglossaryentry{AWMSg}
{
	name=\glslink{AWMS}{AWMS},
	text=Advanced Workforce Management System,
	sort=AWMS,
	description={È una soluzione software che utilizza algoritmi di \gls{machine learning}, per risolvere uno dei problemi cardine di un \gls{plant manager} ovvero, la pianificazione ottimale della forza lavoro che ha disposizione. L'obbiettivo principale della soluzione è quello di pianificare la persona giusta al posto giusto in base alle competenze tecniche possedute del lavoratore.}
}

\newglossaryentry{Electrolux}
{
	name={Electrolux},
	text=Electrolux,
	sort=electrolux,
	description={Electrolux è una multinazionale svedese produttrice di elettrodomestici con sede a Stoccolma}
}
\newglossaryentry{machine learning}
{
	name={Machine learning},
	text=Machine learning,
	sort=machine learning,
	description={Nell'ambito dell'informatica, l'apprendimento automatico è una variante alla programmazione tradizionale nella quale in una macchina si predispone l'abilità di apprendere qualcosa dai dati in maniera autonoma, senza istruzioni esplicite}
}

\newglossaryentry{plant manager}
{
	name={Plant manager},
	text=Plant manager,
	sort=plant manager,
	description={Detto anche responsabile di stabilimento, è colui che presiede e organizza le operazioni quotidiane degli impianti di produzione aziendali, di cui deve assicurare il funzionamento ottimale ed efficiente. Si occupa dei lavoratori, assegnando funzioni e ruoli, definendo orari di lavoro e produzione, formando i neo assunti. Raccoglie e analizza i dati di produzione per trovare eventuali spazi di miglioramento. Si occupa della sicurezza dei lavoratori e quella degli impianti; monitora le apparecchiature di produzione e, in caso di necessità, della loro riparazione o sostituzione}
}

\newglossaryentry{bot}
{
	name={Bot},
	text=Bot,
	sort=bot,
	description={È un software progettato per simulare una conversazione con un essere umano. Lo scopo principale di questi software è quello di simulare un comportamento umano e sono a volte definiti anche agenti intelligenti e vengono usati per vari scopi come la guida in linea, per rispondere alle FAQ degli utenti che accedono a un sito. In alcuni utilizzano sofisticati sistemi di elaborazione del linguaggio naturale, ma molti si limitano a eseguire la scansione delle parole chiave nella finestra di input e fornire una risposta con le parole chiave più corrispondenti}
}

\newglossaryentry{QR code}
{
	name={QR code},
	text=QR code,
	sort=qr-code,
	description={È un codice a barre bidimensionale (o codice 2D), ossia a matrice, composto da moduli neri disposti all'interno di uno schema bianco di forma quadrata, impiegato tipicamente per memorizzare informazioni generalmente destinate a essere lette tramite uno smartphone}
}

\newglossaryentry{WebSocket}
{
	name={WebSocket},
	text=WebSocket,
	sort=websocket,
	description={È una tecnologia web che fornisce canali di comunicazione a due direzioni cioè gli interlecutori possono sia inviare sia ricevere contemporaneamente attraverso una singola connessione TCP}
}

\newglossaryentry{SCRUM}
{
	name={SCRUM},
	text=SCRUM,
	sort=scrum,
	description={È un framework agile per la gestione del ciclo di sviluppo del software, iterativo ed incrementale, concepito per gestire progetti e prodotti software o applicazioni di sviluppo. Nel proprio manifesto prevede i seguenti punti che lo caratterizzano, le persone e le interazioni sono più importanti dei processi e dei strumenti, meglio avere da subito software funzionante che documentazione ampia, meglio una collaborazione con il cliente piuttosto che fare una negoziazione del contratto, essere in grando di rispondere ai cambiamenti piuttosto che rispettare un piano. 
	I progetti Scrum progrediscono attraverso una serie di sprint che hanno una durata massima di un mese. Negli sprint vengono decisi quali requisiti devono essere soddisfatti, e quindi successivamente, progettati, implementati e testati}
}
\begin{comment}
\newglossaryentry{e2eg}
{
	name=\glslink{test e2e}{E2E},
	text=Test End to End,
	sort=test End to End,
	description={Con il termine test end-to-end (end-to-end testing) si intende quell’attività di testing dell’interfaccia grafica vista dagli utenti del programma dall’inizio fino alla fine. In altre parole rappresenta una metodologia utilizzata per verificare se il flusso di un’applicazione si sta comportando come progettato dall’inizio fino alla fine senza che vengano rilevati dei errori che andrebbero a inficiare sulla qualità dell’applicazione stessa}
}
\end{comment}
