
%**************************************************************
% Acronimi
%**************************************************************
\renewcommand{\acronymname}{Acronimi e abbreviazioni}

\newacronym[description={\glslink{apig}{\textit{Application Program Interface}}}]
    {api}{API}{Application Program Interface}
    
\newacronym[description={\textit{Advanced Workforce Management System}}]
	{AWMS}{AWMS}{Advanced Workforce Management System}

\newacronym[description={\glslink{umlg}{\textit{Unified Modeling Language}}}]
    {uml}{UML}{Unified Modeling Language}
    
\newacronym[description={\textit{Test End to End}}]
	{test e2e}{E2E}{test End to End}

\newacronym[description={\textit{Internazionalizzazione}}]
	{i18n}{i18n}{internazionalizzazione}

\newacronym[description={\glslink{JSONg}{\textit{JavaScript Object Notation}}}]
	{JSON}{JSON}{JavaScript Object Notation}
	
\newacronym[description={\glslink{restg}{\textit{Representational State Transfer}}}]
	{rest}{REST}{Representational State Transfer}	

\newacronym[description={\glslink{httpg}{\textit{Hyper Text Transfer Protocol}}}]
{http}{HTTP}{Hyper Text Transfer Protocol}	
	
\newacronym[description={\textit{Hyper Text Markup Language}}]
	{HTML}{HTML}{HyperText Markup Language}	
	
\newacronym[description={\textit{Cascading Style Sheets}}]
	{CSS}{CSS}{Cascading Style Sheets}
	
\newacronym[description={\textit{Syntactically Awesome StyleSheets}}]
	{Sass}{Sass}{Syntactically Awesome StyleSheets}
	
\newacronym[description={\textit{Hypertext Preprocessor}}]
	{PHP}{PHP}{Hypertext Preprocessor}	
	
\newacronym[description={\glslink{urlg}{\textit{Uniform Resource Locator}}}]
	{url}{URL}{Uniform Resource Locator}	
			
\newacronym[description={\glslink{httpsg}{\textit{Hyper Text Transfer Protocol over Secure Socket Layer}}}]
	{https}{HTTPS}{Hyper Text Transfer Protocol over Secure Socket Layer}	
		
\newacronym[description={\glslink{dbmsg}{\textit{Database Management System}}}]
	{DBMS}{DBMS}{Database Management System}

\newacronym[description={\glslink{domg}{\textit{Document Object Model}}}]
	{DOM}{DOM}{Document Object Model}

\newacronym[description={\textit{World Wide Web Consortium}}]
	{W3C}{W3C}{World Wide Web Consortium}

\newacronym[description={\glslink{gdprg}{\textit{General Data Protection Regulation}}}]
	{GDPR}{GDPR}{General Data Protection Regulation}	

\newacronym[description={\glslink{sdkg}{\textit{Software Development Kit}}}]
	{sdk}{SDK}{Software Development Kit}

\newacronym[description={\textit{eXtensible Markup Language}}]
	{XML}{XML}{eXtensible Markup Language}	

\newacronym[description={\textit{eXtensible Hyper Text Markup Language}}]
	{XHTML}{XHTML}{eXtensible Hyper Text Markup Language}
	
\newacronym[description={\textit{behavior-driven development}}]
	{BDD}{BDD}{behavior-driven development}
%**************************************************************
% Glossario
%**************************************************************
\renewcommand{\glossaryname}{Glossario}

\newglossaryentry{apig}
{
	name=\glslink{api}{\textit{API}},
	text=Application Program Interface,
	sort=api,
	description={In informatica con il termine \emph{Application Programming Interface (API)} (ing. interfaccia di programmazione di un'applicazione) si indica ogni insieme di procedure disponibili al programmatore, di solito raggruppate a formare un set di strumenti specifici per l'espletamento di un determinato compito all'interno di un certo programma. La finalità è ottenere un'astrazione, di solito tra l'\emph{hardware} e il programmatore o tra \emph{software} a basso e quello ad alto livello semplificando così il lavoro di programmazione}
}

\newglossaryentry{umlg}
{
	name=\glslink{uml}{\textit{UML}},
	text=UML,
	sort=uml,
	description={in ingegneria del software \emph{UML, Unified Modeling Language} (ing. linguaggio di modellazione unificato) è un linguaggio di modellazione e specifica basato sul paradigma object-oriented. L'\emph{UML} svolge un'importantissima funzione di ``lingua franca'' nella comunità della progettazione e programmazione a oggetti. Gran parte della letteratura di settore usa tale linguaggio per descrivere soluzioni analitiche e progettuali in modo sintetico e comprensibile a un vasto pubblico}
}

\newglossaryentry{Electrolux}
{
	name={\textit{Electrolux}},
	text=Electrolux,
	sort=electrolux,
	description={Electrolux è una multinazionale svedese produttrice di elettrodomestici con sede a Stoccolma}
}

\newglossaryentry{machine learning}
{
	name={\textit{Machine learning}},
	text=machine learning,
	sort=machine learning,
	description={Nell'ambito dell'informatica, (ing.apprendimento automatico) l'apprendimento automatico è una variante alla programmazione tradizionale nella quale in una macchina si predispone l'abilità di apprendere qualcosa dai dati in maniera autonoma, senza istruzioni esplicite}
}

\newglossaryentry{plant manager}
{
	name={\textit{Plant manager}},
	text=plant manager,
	sort=plant manager,
	description={È colui (ing.responsabile di stabilimento) che presiede e organizza le operazioni quotidiane degli impianti di produzione aziendali, di cui deve assicurare il funzionamento ottimale ed efficiente. Si occupa dei lavoratori, assegnando funzioni e ruoli, definendo orari di lavoro e produzione. Raccoglie e analizza i dati di produzione per trovare eventuali spazi di miglioramento. Si occupa della sicurezza dei lavoratori e quella degli impianti inoltre, monitora le apparecchiature di produzione e, in caso di necessità, della loro riparazione o sostituzione}
}

\newglossaryentry{bot}
{
	name={\textit{Bot}},
	text=bot,
	sort=bot,
	description={È un \emph{software} progettato per simulare una conversazione con un essere umano. Lo scopo principale di questi \emph{software} è quello di simulare un comportamento umano e sono a volte definiti anche agenti intelligenti e vengono usati per vari scopi come la guida in linea, per rispondere alle FAQ degli utenti che accedono a un sito. In alcuni utilizzano sofisticati sistemi di elaborazione del linguaggio naturale, ma molti si limitano a eseguire la scansione delle parole chiave nella finestra di input e fornire una risposta con le parole chiave più corrispondenti}
}

\newglossaryentry{QR code}
{
	name={\textit{QR code}},
	text=QR code,
	sort=qr-code,
	description={È un codice a barre bidimensionale (o codice 2D), ossia a matrice, composto da moduli neri disposti all'interno di uno schema bianco di forma quadrata, impiegato tipicamente per memorizzare informazioni generalmente destinate a essere lette tramite uno \emph{smartphone}}
}

\newglossaryentry{WebSocket}
{
	name={\textit{WebSocket}},
	text=WebSocket,
	sort=websocket,
	description={È una tecnologia web che fornisce canali di comunicazione a due direzioni cioè gli interlecutori possono sia inviare sia ricevere contemporaneamente attraverso una singola connessione TCP}
}

\newglossaryentry{SCRUM}
{
	name={\textit{SCRUM}},
	text=SCRUM,
	sort=scrum,
	description={È un \emph{framework} agile per la gestione del ciclo di sviluppo del \emph{software}, iterativo ed incrementale, concepito per gestire progetti e prodotti software o applicazioni di sviluppo. Nel proprio manifesto prevede i seguenti punti che lo caratterizzano, le persone e le interazioni sono più importanti dei processi e degli strumenti, meglio avere da subito \emph{software} funzionante che documentazione ampia, meglio una collaborazione con il cliente piuttosto che fare una negoziazione del contratto, essere in grando di rispondere ai cambiamenti piuttosto che rispettare un piano. 
		I progetti Scrum progrediscono attraverso una serie di sprint che hanno una durata massima di un mese. Negli sprint vengono decisi quali requisiti devono essere soddisfatti, e quindi successivamente, progettati, implementati e testati}
}

\newglossaryentry{framework}{
	name={\textit{Framework}},
	text=framework,
	sort=framework,
	description={In informatica con il termine framework 
		si indica un insieme di elementi \emph{software} che un programmatore può usare o modificare per realizzare un programma. Rappresenta un’astrazione composta da elementi universali e riutilizzabili con lo scopo di facilitare lo sviluppo di un programma e di far applicare buone norme di programmazione. Inoltre, un framework può offrire programmi di supporto, librerie, compilatori e documentazione per l'utilizzo}
}

\newglossaryentry{iOS}{
	name={\textit{iOS}},
	text=iOS,
	sort=iOS,
	description={È un sistema operativo \emph{mobile} sviluppato da Apple per iPhone, iPod touch e iPad. Le versioni principali di iOS vengono distribuite ogni anno. L'attuale versione, iOS 13, è stata distribuita al pubblico il 19 settembre 2019}
}

\newglossaryentry{Android}{
	name={\textit{Android}},
	text=Android,
	sort=Android,
	description={È un sistema operativo per dispositivi \emph{mobile} sviluppato da Google e basato sul kernel Linux, progettato principalmente per \emph{smartphone} e \emph{tablet}, interfacce utente specializzate per televisori (Android TV), automobili (Android Auto), orologi da polso (Wear OS), occhiali (Google Glass). L'attuale ultima versione è Android 11}
}

\newglossaryentry{architettura}{
	name={\textit{Architettura}},
	text=architettura,
	sort=architettura,
	description={In informatica con il termine architettura, in questo caso intesa come architettura \emph{software}, è l'organizzazione fondamentale di un sistema, definita dai suoi componenti, dalle relazioni reciproche tra i componenti e con l'ambiente, e i principi che ne governano la progettazione e l'evoluzione}
}

\newglossaryentry{linguaggio di markup}{
	name={\textit{Linguaggio di markup}},
	text=linguaggio di markup,
	sort=linguaggio di markup,
	description={In informatica con il termine linguaggio di markup, si intende un gruppo di regole detti marcatori, attraverso le quali vengono descritti i meccanismi di rappresentazione di un testo}
}

\newglossaryentry{browser web}{
	name={\textit{Browser web}},
	text=browser web,
	sort=browser web,
	description={In informatica si intende un'applicazione per l'acquisizione, la presentazione e la navigazione di risorse sul web. Permette la visualizzazione dei contenuti ipertestuali, e la riproduzione di contenuti multimediali. Tra i browser più popolari vi sono Google Chrome, Internet Explorer, Mozilla Firefox, Microsoft Edge, Safari, Opera}
}

\newglossaryentry{applicazione nativa}{
	name={\textit{Applicazione nativa}},
	text=applicazione nativa,
	sort=applicazione nativa,
	description={In informatica si intende un'applicazione scritta e compilata per una specifica piattaforma utilizzando i linguaggi di programmazione e librerie supportati dal particolare sistema operativo \emph{mobile}}
}

\newglossaryentry{applicazione web mobile}{
	name={\textit{Applicazione web mobile}},
	text=applicazione web mobile,
	sort=applicazione web mobile,
	description={In informatica si intende pagine web ottimizzate per dispositivi \emph{mobile} scritte utilizzando tecnologie web, in particolare \gls{HTML}, JavaScript e \gls{CSS}. Inoltre, le applicazioni web non possono accedere alle funzionalità del dispositivo ad'esempio la fotocamera}
}

\newglossaryentry{applicazione ibrida}{ %va a benzina ma anche con l'elettricità
	name={\textit{Applicazione ibrida}},
	text=applicazione ibrida,
	sort=applicazione ibrida,
	description={In informatica si intende applicazioni sviluppate con tecnologie web e vengono eseguite localmente all’interno di un’applicazione nativa. Grazie a ciò possono interagire con il dispositivo ad'esempio utilizzare la fotocamera}
}

\newglossaryentry{notifica push}{ 
	name={\textit{Notifica push}},
	text=notifica push,
	sort=notifica push,
	description={In informatica si intende un tipo di messaggistica istantanea grazie alla quale il messaggio perviene al destinatario senza che questo debba effettuare un'operazione di scaricamento. Tale modalità è quella tipicamente usata da applicazioni come \emph{WhatsApp} o da servizi di sistemi operativi come \g{Android}, oppure da numerose applicazioni derivate da siti web come, ad esempio, il servizio meteo o quello delle notizie}
}

\newglossaryentry{licenza MIT}{ 
	name={\textit{Licenza MIT}},
	text=licenza MIT,
	sort=licenza MIT,
	description={La Licenza MIT è una licenza di \emph{software} libero. È una licenza permissiva, cioè permette il riutilizzo nel \emph{software} proprietario sotto la condizione che la licenza sia distribuita con tale \emph{software}}
}

\newglossaryentry{JSONg}{ 
	name=\glslink{JSON}{\textit{JSON}},
	text=JavaScript Object Notation,
	sort=JSON,
	description={In informatica con il termine \emph{JavaScript Object Notation (JSON)}(ing.Notazione degli oggetti JavaScript) si intende un formato testuale standard, usato per rappresentare dati strutturati basati sulla sintassi degli oggetti in JavaScript. È comunemente utilizzato per l'interscambio di dati fra applicazioni client/server. Risulta essere facile da comprendere e da scrivere per le persone mentre per le macchine risulta essere un formato leggero e veloce da analizzare}
}

\newglossaryentry{pooling}{ 
	name={\textit{Pooling}},
	text=pooling,
	sort=pooling,
	description={In informatica con il termine pooling si intende una procedura attraverso la quale periodicamente viene eseguita una operazione. Nel caso delle comunicazioni tra \g{client} e \g{server} il pooling è la richiesta periodica del \g{client} di dati al \g{server} per controllare sei i dati che ha sono aggiornati}
}

\newglossaryentry{client}{ 
	name={\textit{Client}},
	text=client,
	sort=client,
	description={In informatica con il termine client si intende un entita presente in un rete di comunicazione che accede ai servizi o alle risorse messe a dispozione da un'altra componente detta \g{server}, la cui comunicazione tra client e \g{server} è regolata da insieme di regole e norme detti protocolli di comunicazione. Insieme al \g{server} forma l'architettura client/server}
}

\newglossaryentry{server}{ 
	name={\textit{Server}},
	text=server,
	sort=server,
	description={In informatica con il termine client si intende un entita presente in un rete di comunicazione che offre dei servizi o dalle risorse a un'altri componenti presenti nella rete detti \g{client}, la cui comunicazione tra \g{client} e server è regolata da insieme di regole e norme detti protocolli di comunicazione. Insieme al \g{client} forma l'architettura client/server}
}

\newglossaryentry{httpg}{ 
	name=\glslink{http}{\textit{HTTP}},
	text=Hyper Text Transfer Protocol,
	sort=http,
	description={In informatica con il termine \emph{Hyper Text Transfer Protocol (HTTP)} (ing. protocollo di trasferimento di un ipertesto) si intende un insieme di regole e norme che regolano la trasmissione e la comunicazione d'informazione nella rete Internet. Questo trasmissione d'informazioni avviene sotto forma di scambi di messaggi tipicamente tra il client che puo essere un \g{browser web}, e un server}
}

\newglossaryentry{open-source}{ 
	name={\textit{Open-source}},
	text=open-source,
	sort=open-source,
	description={In informatica con il termine open-source (ing. sorgente libero) si intende un \emph{software} per cui chi lo ha sviluppato rinuncia alla propretà del \emph{software} dando libero accesso a tutto il codice sorgente a chiunque, e quindi è permesso a tutti di contribuire nello sviluppo del codice al fine di migliorarlo, aggiungere nuove funzionalità o  correggere errori all'interno del codice}
}

\newglossaryentry{restg}{ 
	name=\glslink{rest}{\textit{REST}},
	text=Representational State Transfer,
	sort=rest,
	description={In informatica con il termine \emph{Representational State Transfer (REST)} (ing. trasferimento di stato rappresentativo) si intende un approccio architetturale alla creazione di web \g{api} basato sul protocollo di comunicazione \g{http}. Viene imposto che le \g{api} devono permettere di accedere a delle risorse attraverso un \g{url}, utilizzare il formato \gls{JSON} e \gls{XML}, non avere uno stato cioè non deve essere memorizzato cioè che è stato fatto e infine, utilizzare i metodi del \g{http}, GET, POST, PUT, DELETE}
}

\newglossaryentry{urlg}{ 
	name=\glslink{url}{\textit{URL}},
	text=Uniform Resource Locator,
	sort=url,
	description={In informatica con il termine \emph{Uniform Resource Locator (URL)} (ing. localizzare di risorse uniformi) in intede una sequenza di caratteri che identifica univocamente l'indirizzo di una risorsa su una rete di computer, come ad esempio un documento, un'immagine, un video, tipicamente presente su un \g{server} e resa accessibile a un \g{client}}
}

\newglossaryentry{front-end}{ 
	name={\textit{Front-end}},
	text=front-end,
	sort=front-end,
	description={In informatica con il termine front-end si intende la parte visibile all'utente di un programma e con cui egli può interagire solitamente è un'interfaccia utente. Perciò il front-end è la parte di un sistema \emph{software} che gestisce l'interazione con l'utente, ricevendo da esso un input da cui viene prodotto (dal \g{back-end}) un output da mostrare all'utente}
}

\newglossaryentry{back-end}{ 
	name={\textit{Back-end}},
	text=back-end,
	sort=back-end,
	description={In informatica con il termine back-end si intende la parte che si occuppa di ricevere in input dati inseriti dall'utenti e di eleborarli per rispondere alle richieste dell'utente. Dopo l'elaborazione dei dati il back-end produce un risultato che sarà compito del \g{front-end} mostrarlo}
}

\newglossaryentry{httpsg}{ 
	name=\glslink{https}{\textit{HTTPS}},
	text=Hyper Text Transfer Protocol over Secure Socket Layer,
	sort=https,
	description={In informatica con il termine \emph{Hyper Text Transfer Protocol over Secure Socket Layer (HTTPS)} (ing. protocollo di trasferimento di un ipertesto basato su un strato di sicurrezza) si intende un insieme di regole e norme che regolano la trasmissione e la comunicazione d'informazione nella rete Internet in modo sicuro cioè il contenuto della trasmissione non è interpretabile da entità diverse dal mittente o dal/dai destinataro/i.
		Per la comunicazione viene utilizzato il protocollo \g{http} all'interno di una connessione criptata dal protocollo \emph{Secure Sockets Layer (SSL)} garantendo così riservatezza dei dati cioè il contenuto della tramissione e visibile solo al mittente e al destinatario, integrità dei dati cioè il contenuto della trasmessione non viene alterato e autenticazione di comunica}
}

\newglossaryentry{database}{ 
	name={\textit{Database}},
	text=database,
	sort=database,
	description={In informatica con il termine database (ing. base di dati) si intende una collezione di dati ben organizzati e ben strutturati, gestiti in modo integrato da un sistema per la gestione delle basi di dati, costituiscono una base di lavoro per utenti diversi con programmi diversi. I prodotti \emph{software} per la gestione dei database sono indicati con il termine \g{DBMS}}
}

\newglossaryentry{dbmsg}{ 
	name=\glslink{DBMS}{\textit{DBMS}},
	text=Database Management System,
	sort=dbms,
	description={In informatica con il termine \emph{Database Management System (DBMS)} (ing. sistema di gestione di basi di dati) in intede un sistema \emph{software} progettato per consentire la creazione, la manipolazione e l'interrogazione efficiente di \g{database}}
}

\newglossaryentry{firebase}{ 
	name={\textit{Firebase}},
	text=firebase,
	sort=firebase,
	description={È la piattaforma \emph{mobile} di Google che aiuta nello sviluppare applicazione \emph{mobile}. Firebase offre tutto ciò che dovrebbe offrire un \g{back-end} quindi, funzionalità di autenticazione, un \g{database} (quello di Firebase è di tipo NoSQL), servizi di \emph{hosting} e algoritmi di \g{machine learning} per l'apprendimento automatico}
}

\newglossaryentry{foreground}{ 
	name={\textit{Esecuzione in foreground}},
	text=foreground,
	sort=foreground,
	description={In informatica con il termine foreground (ing. primo piano) si intende l'esecuzione dei processi di un \emph{software} dove può essere richiesta l'interazione dell'utente ma che comunque l'utente sa dell'esecuzioni di tali processi. Nel caso di un'applicazione \emph{mobile} significa che nello schermo viene visualizzata l'applicazione in esecuzione con la quale l'utente può interagire}
}

\newglossaryentry{background}{ 
	name={\textit{Esecusione in background}},
	text=background,
	sort=background,
	description={In informatica con il termine background, si intede l'esecuzione dei processi di un \emph{software} dove non viene richiesto l'intervento dell'utente, tanto da non essere a lui visibile tale esecuzione. Nel caso di un'applicazione \emph{mobile} significa che nello schermo non viene visualizzata l'applicazione in esecuzione. Resta comunque attiva ma non interagibile con l'utente finché è in background}
}

\newglossaryentry{base64}{ 
	name={\textit{Base64}},
	text=base64,
	sort=base64,
	description={In informatica con il termine base64, si intede un sistema di codifica che consente la traduzione di dati binari in stringhe di testo ASCII cioè un insieme di codici per la codifica dei caratteri. I dati vengono rappresentati sulla base di sessantaquattro caratteri ASCII diversi}
}

\newglossaryentry{domg}{ 
	name=\glslink{DOM}{\textit{DOM}},
	text=Document Object Model,
	sort=dom,
	description={In informatica con il termine \emph{Document Object Model (DOM)} (ing. modello a oggetti del documento) in intede una forma di rappresentazione dei documenti strutturati in modo gerarchico. È lo standard ufficiale del \gls{W3C} per la rappresentazione di documenti strutturati in maniera da essere neutrali sia per la lingua che per la piattaforma}
}

\newglossaryentry{gdprg}{ 
	name=\glslink{GDPR}{\textit{GDPR}},
	text=General Data Protection Regulation,
	sort=gdpr,
	description={Per \emph{General Data Protection Regulation (GDPR)} (ing. Regolamento generale sulla protezione dei dati) in intede un regolamento dell'Unione europea in materia di trattamento dei dati personali e di \emph{privacy}, adottato il 27 aprile 2016, pubblicato sulla Gazzetta ufficiale dell'Unione europea il 4 maggio 2016 ed entrato in vigore il 24 maggio dello stesso anno ed operativo a partire dal 25 maggio 2018. Il testo affronta anche il tema dell'esportazione di dati personali al di fuori dell'UE e obbliga tutti i titolari del trattamento dei dati (anche con sede legale fuori dall'UE) che trattano dati di residenti nell'UE ad osservare e adempiere agli obblighi previsti.}
}

\newglossaryentry{sdkg}
{
	name=\glslink{sdk}{\textit{SDK}},
	text=Software Development Kit,
	sort=sdk,
	description={in informatica con il termine \emph{Software Development Kit (SDK)} (ing. pacchetto di sviluppo per \textit{software}) si intende una collezione di strumenti per lo sviluppo \emph{software} contenuti all'interno di un singolo pacchetto installabile all'interno del proprio sistema. Tutto ciò viene offerto per facilitare la creazione di applicazioni. Questi strumenti solitamente sono specifici per il particolare tipo di \emph{hardware}, sistema operativo e linguaggio di programmazione utilizzati per lo sviluppo \emph{software}}
}

\newglossaryentry{mock}{ 
	name={\textit{Mock}},
	text=mock,
	sort=mock,
	description={In informatica con il termine mock, si intede un oggetto che cerca di ripordure il comportamento di un oggetto reale in modo controllato, con l'obbiettivo di testare il comportamento di altri oggetti reali che dipendono dall'oggetto che si sta simulando con il mock}
}

\newglossaryentry{design pattern}{ 
	name={\textit{Design pattern}},
	text=design pattern,
	sort=design pattern,
	description={In informatica e specialmente nell'ambito dell'Ingegneria del Software con il termine design pattern, si intede di una descrizione o modello logico da applicare per la risoluzione di un problema che può presentarsi in diverse situazioni durante le fasi di progettazione e sviluppo del \emph{software}, ancor prima della definizione dell'algoritmo risolutivo della parte computazionale. È un approccio spesso efficace nel contenere o ridurre i costi per lo sviluppo del \emph{software}}
}


\begin{comment}
\newglossaryentry{e2eg}
{
	name=\glslink{test e2e}{E2E},
	text=Test End to End,
	sort=test End to End,
	description={Con il termine test end-to-end (end-to-end testing) si intende quell’attività di testing dell’interfaccia grafica vista dagli utenti del programma dall’inizio fino alla fine. In altre parole rappresenta una metodologia utilizzata per verificare se il flusso di un’applicazione si sta comportando come progettato dall’inizio fino alla fine senza che vengano rilevati dei errori che andrebbero a inficiare sulla qualità dell’applicazione stessa}
}
\newglossaryentry{AWMSg}
{
	name=\glslink{AWMS}{AWMS},
	text=Advanced Workforce Management System,
	sort=AWMS,
	description={È una soluzione software che utilizza algoritmi di \gls{machine learning}, per risolvere uno dei problemi cardine di un \gls{plant manager} ovvero, la pianificazione ottimale della forza lavoro che ha disposizione. L'obbiettivo principale della soluzione è quello di pianificare la persona giusta al posto giusto in base alle competenze tecniche possedute del lavoratore}
}

Parole da aggiungere


\end{comment}
